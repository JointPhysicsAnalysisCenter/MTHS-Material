
\newcommand{\ma}{\ensuremath{M}\xspace}
\newcommand{\mb}{\ensuremath{m_1}\xspace}
\newcommand{\mc}{\ensuremath{m_2}\xspace}
\newcommand{\eb}{\ensuremath{E_1}\xspace}
\newcommand{\ec}{\ensuremath{E_2}\xspace}

\newcommand{\pb}{\ensuremath{p_1}\xspace}
\newcommand{\pc}{\ensuremath{p_2}\xspace}
\newcommand{\pvecb}{\ensuremath{\textbf{p}_1}\xspace}
\newcommand{\pvecc}{\ensuremath{\textbf{p}_2}\xspace}
\newcommand{\pvecstar}{\ensuremath{\textbf{p}^{*}}\xspace}
\newcommand{\pmodvecb}{\ensuremath{\text{p}_1}\xspace}
\newcommand{\pmodvecc}{\ensuremath{\text{p}_2}\xspace}
\newcommand{\pmodvecstar}{\ensuremath{\text{p}^{*}}\xspace}


In hadron physics, the Mandelstam invariants \(s\), \(t\), and \(u\) are essential for describing the kinematics of scattering processes. Consider a \(2 \rightarrow 2\) scattering process where two particles with four-momenta \(p_1\) and \(p_2\) scatter into two particles with four-momenta \(p_3\) and \(p_4\). Consider the particles have different masses. The Mandelstam invariants are defined as \(s = (p_1 + p_2)^2\), \(t = (p_1 - p_3)^2\), and \(u = (p_1 - p_4)^2\). 
The scattering angle \(\theta\) is the angle between the momenta of the incoming and outgoing particles in the center of mass frame. 
%In the center of mass frame, the total energy is given by \(E_{\text{cm}}\), and the momenta of the incoming particles are equal and opposite. The four-momenta can be expressed as \(p_1 = (E_{\text{cm}}/2, \mathbf{p})\) and \(p_2 = (E_{\text{cm}}/2, -\mathbf{p})\), with the final state momenta \(p_3 = (E_{\text{cm}}/2, \mathbf{p'})\) and \(p_4 = (E_{\text{cm}}/2, -\mathbf{p'})\). Using the scattering angle \(\theta\), relate the magnitudes of the momenta \(\mathbf{p}\) and \(\mathbf{p'}\). Derive the expressions for the Mandelstam invariants \(s\), \(t\), and \(u\) in terms of \(E_{\text{cm}}\) and \(\theta\). 

%Furthermore, express the breakup momentum in the rest frame, the scattering angle, and the energy of the particles in the rest frame of the system in terms of the Mandelstam variables. The breakup momentum \(p\) in the center of mass frame can be defined as the magnitude of the momentum of either particle in this frame. 


\begin{enumerate}
\item Verify the relation \(s + t + u = \sum m_i^2\), where \(m_i\) are the masses of the particles.
\item Calculate the particle energies and 3-momenta in the center of mass frame, as a function of $s$
    \item Calculate the scattering angle  \(\theta\) in terms of \(s\), \(t\), and \(u\)
    \item How would those relation change if one considered the crossed reaction $p_1 p_{\bar 3} \to p_{\bar 2} p_4$?
\end{enumerate}
