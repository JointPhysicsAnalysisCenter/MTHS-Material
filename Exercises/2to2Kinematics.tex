
\newcommand{\ma}{\ensuremath{M}\xspace}
\newcommand{\mb}{\ensuremath{m_1}\xspace}
\newcommand{\mc}{\ensuremath{m_2}\xspace}
\newcommand{\eb}{\ensuremath{E_1}\xspace}
\newcommand{\ec}{\ensuremath{E_2}\xspace}

\newcommand{\pb}{\ensuremath{p_1}\xspace}
\newcommand{\pc}{\ensuremath{p_2}\xspace}
\newcommand{\pvecb}{\ensuremath{\textbf{p}_1}\xspace}
\newcommand{\pvecc}{\ensuremath{\textbf{p}_2}\xspace}
\newcommand{\pvecstar}{\ensuremath{\textbf{p}^{*}}\xspace}
\newcommand{\pmodvecb}{\ensuremath{\text{p}_1}\xspace}
\newcommand{\pmodvecc}{\ensuremath{\text{p}_2}\xspace}
\newcommand{\pmodvecstar}{\ensuremath{\text{p}^{*}}\xspace}


In hadron physics, the Mandelstam invariants $s$, $t$, and $u$ are essential for describing the kinematics of scattering processes.
Consider a $2 \rightarrow 2$ scattering process where two particles with four-momenta $p_1$ and $p_2$ scatter into two particles with four-momenta $p_3$ and $p_4$.
Consider the particles have different masses. The Mandelstam invariants are defined as $s = (p_1 + p_2)^2$, $t = (p_1 - p_3)^2$, and $u = (p_1 - p_4)^2$.
The scattering angle $\theta$ is the angle between the momenta of the incoming and outgoing particles in the center of mass frame.

\begin{enumerate}
    \item Verify the relation $s + t + u = \sum m_i^2$, where $m_i$ are the masses of the particles.
    \item Calculate the particle energies and 3-momenta in the center of mass frame, as a function of $s$
    \item Calculate the scattering angle  $\theta$ in terms of $s$, $t$, and $u$
    \item How would those relation change if one considered the crossed reaction $p_1 p_{\bar 3} \to p_{\bar 2} p_4$?
\end{enumerate}
