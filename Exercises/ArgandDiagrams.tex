
The $\pi\pi$ scattering with
unphysical pion mass ($m_\pi=391\,$MeV) for S (left) and D (right) partial waves
is studied using \href{https://inspirehep.net/literature/1618009}{lattice calculations}.
Scattering amplitudes are presented on the Argand diagrams
(parametric plot of energy in Real/Imaginary coordinates) as a function of energy of the system.
The value are given units of $E_\text{cm} \cdot t$ where $t \cdot m_\pi$ = 0.06906.

\twofig{0.48}{Swave_ArgandDiagram}{0.48}{Dwave_ArgandDiagram}

Using information on the diagrams, answer the following questions:
\be
\item Estimate masses of $K$ and $\eta$ particles.
\item Find the elastic energy region for the S and D waves.

\begin{solution}
	Elastic region is defined as the range of energy values for which $\pi\pi \rightarrow \pi\pi$ process dominates. For the S-wave, the elastic region lies for the $E_\text{cm}t$ range of [0.139,0.189] (after this point, the amplitude hits the $K\bar{K}$ threshold. Also, after this point, the curve starts going inside the unitarity circle). For the D-wave, this region exists until value of 0.229.
\end{solution}
\item Locate the energy value for which the S-wave peak.

\begin{solution}
	The S-wave peaks at the point $E_\text{cm}t$ = 0.154, which is at energy $E_\text{cm} = 0.872$ GeV.
\end{solution}
\item Estimate the mass and decay width for the D wave resonance.

\begin{solution}
	The D-wave resonance is observed at $E_\text{cm}t = 0.284$, this is the point where there is a kink in the argand diagram. \textit{Note : I can locate where the resonance is. I am confused about how to proceed from there, because I can get center of mass energy from the point, and maybe equate it to the pole value. But I would expect a complex output but I cannot read it out properly from the Argand diagram}.
\end{solution}

\item Sketch the amplitude phase versus energy of the system for both partial waves.
\ee