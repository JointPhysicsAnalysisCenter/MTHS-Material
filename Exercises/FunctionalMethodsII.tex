

Now think about a four-quark state with two heavy quarks and two light anti-quarks in the two 
flavour combinations $bb\bar{q}\bar{q}$ and $bc\bar{q}\bar{q}$. Suppose, the quarks and antiquarks are 
arranged in scalar (S) and axialvector (A) diquarks. Which diquark combinations are possible for the following quantum numbers?

\begin{enumerate}
	\item $I(J) = 0(1)$
	      
	      \begin{solution}
		      $J=1 \rightarrow$ we need at least one axialvector diquark, i.e. only combinations $AA, SA, AS$ are possible.\\ 
		      \underline{color $3 \bigotimes \bar{3}$:}\\
		      $I=0 \rightarrow$ light diquark needs to be S $\rightarrow$ heavy diquark needs to be A and indeed, this is possible for
		      both, $bb\bar{q}\bar{q}$ and $bc\bar{q}\bar{q}$.\\
		      \underline{color $6 \bigotimes \bar{6}$:}\\
		      $I=0 \rightarrow$ light diquark needs to be A $\rightarrow$ heavy diquark can be S or A. Only S is possible for $bb\bar{q}\bar{q}$, whereas S and A are possible for $bc\bar{q}\bar{q}$. 	
	      \end{solution}
	      
	\item $I(J) = 1(1)$
	      
	      \begin{solution}
		      $J=1 \rightarrow$ we need at least one axialvector diquark, i.e. only combinations $AA, SA, AS$ are possible.\\ 
		      \underline{color $3 \bigotimes \bar{3}$:}\\
		      $I=1 \rightarrow$ light diquark needs to be A. For $bb\bar{q}\bar{q}$, the heavy diquark also needs to be A 
		      (since S is not possible). For $bc\bar{q}\bar{q}$ both,
		      S and A are possible.\\
		      \underline{color $6 \bigotimes \bar{6}$:}\\
		      $I=1 \rightarrow$ light diquark needs to be S $\rightarrow$ heavy diquark needs to be A. This is not possible for
		      $bb\bar{q}\bar{q}$, but allowed for $bc\bar{q}\bar{q}$.		
	      \end{solution}
	      
	\item $I(J) = 0(0)$
	      
	      \begin{solution}
		      $J=0 \rightarrow$ we need either $SS$ or $AA$ (from rules of adding angular momenta).\\ 
		      \underline{color $3 \bigotimes \bar{3}$:}\\
		      $I=0 \rightarrow$ light diquark needs to be S $\rightarrow$ heavy diquark also needs to be S. This is not possible
		      for $bb\bar{q}\bar{q}$, but allowed for $bc\bar{q}\bar{q}$.\\
		      \underline{color $6 \bigotimes \bar{6}$:}\\
		      $I=0 \rightarrow$ light diquark needs to be A $\rightarrow$ heavy diquark also needs to be A. Again, this is not possible for $bb\bar{q}\bar{q}$, but allowed for $bc\bar{q}\bar{q}$.		
	      \end{solution}
	      
	      {\em Hint: again carefully think about symmetries...}
	      
	      
	      
\end{enumerate}