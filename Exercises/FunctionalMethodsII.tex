

	Now think about a four-quark state with two heavy quarks and two light anti-quarks in the two 
	flavour combinations $bb\bar{q}\bar{q}$ and $bc\bar{q}\bar{q}$. Suppose, the quarks and antiquarks are 
	arranged in scalar (S) and axialvector (A) diquarks. Which diquark combinations are possible for the following quantum numbers?

\begin{enumerate}
	\item $I(J) = 0(1)$ 
	
	\begin{solution}
		$J=1 \rightarrow$ we need at least one axialvector diquark, i.e. only combinations $AA, SA, AS$ are possible.\\ 
		\underline{color $3 \bigotimes \bar{3}$:}\\
		$I=0 \rightarrow$ light diquark needs to be S $\rightarrow$ heavy diquark needs to be A and indeed, this is possible\\
		\underline{color $6 \bigotimes \bar{6}$:}\\
        $I=0 \rightarrow$ light diquark needs to be A $\rightarrow$ heavy diquark needs to be S and indeed, this is possible		
    \end{solution}

	\item $I(J) = 1(1)$ 
	
	\begin{solution}
		$J=1 \rightarrow$ we need at least one axialvector diquark, i.e. only combinations $AA, SA, AS$ are possible.\\ 
        \underline{color $3 \bigotimes \bar{3}$:}\\
        $I=1 \rightarrow$ light diquark needs to be A. Heavy diquark also needs to be A (S is not possible).\\
        \underline{color $6 \bigotimes \bar{6}$:}\\
        $I=1 \rightarrow$ light diquark needs to be S $\rightarrow$ heavy diquark needs to be A, but this is not possible.		
    \end{solution}

	\item $I(J) = 0(0)$ 
	
	\begin{solution}
		$J=0 \rightarrow$ we need either $SS$ or $AA$ (from rules of adding angular momenta).\\ 
        \underline{color $3 \bigotimes \bar{3}$:}\\
        $I=0 \rightarrow$ light diquark needs to be S $\rightarrow$ heavy diquark also needs to be S, but this is not possible.\\
        \underline{color $6 \bigotimes \bar{6}$:}\\
        $I=0 \rightarrow$ light diquark needs to be A $\rightarrow$ heavy diquark also needs to be A, but this is not possible.		
    \end{solution}
	      
	{\em Hint: again carefully think about symmetries...}
	
	
	
\end{enumerate}