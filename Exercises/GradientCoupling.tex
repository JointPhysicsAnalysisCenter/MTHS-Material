
Consider the pseudo-vector (or gradient coupling) to the nucleon described by the Lagrangian
\begin{equation}
    \mathcal{L}=-\frac{f_{\pi NN}}{m_{\pi}} \bar \psi \gamma^{\mu} \gamma_5 \vec{\tau} \psi \cdot \partial_{\mu} \vec{\phi}^{(\pi)} ,
\end{equation}
and {\bf compute the contribution of the following diagram to the one-pion exchange potential (OPEP)}
\fig{0.3}{diagram.png}

Some hints:
\begin{itemize}
    \item You should compute $\bar u (p_1',s_1) \Gamma_{\pi NN} u(p_1,s_1)$, with $\Gamma_{\pi NN}= (i)^2 \frac{f_{\pi NN}}{m_{\pi}} \gamma^{\mu}  \gamma_5 \vec{\tau} q_{\mu}$ for the incoming pion.
          Note that $i$ is the imaginary unit, ($\gamma^{\mu},  \gamma_5$) are  the gamma matrices, $\vec{\tau}$ is  the isospin vector,  $q$ is the four-momentum carried by the pion ($q_{\mu}=p_1'-p_1$),  ${f_{\pi NN}}$ is the ${\pi NN}$ coupling and $m_{\pi}$ is the pion mass.
    \item Consider the static limit ($q_0 \rightarrow 0$)
    \item The Dirac spinors $u(p,s)$ in the non-relativistic approach are given by $u(p,s)=\left( \begin{array} {c}
                  \chi_s \\
                  0
              \end{array} \right) ,$
          with $\chi_s$ the two-component Pauli spinor.
    \item The gamma matrices are defined as
          \begin{equation}
              \gamma^0=
              \left( \begin{array} {cc}
                  \mathbb{1} & 0           \\
                  0          & -\mathbb{1}
              \end{array} \right),\quad
              %
              \gamma^k=
              \left( \begin{array} {cc}
                  0         & \sigma^k \\
                  -\sigma^k & 0
              \end{array} \right) ,\quad
              %
              \gamma^5=\gamma_5=
              \left( \begin{array} {cc}
                  0          & \mathbb{1} \\
                  \mathbb{1} & 0
              \end{array} \right)
              \nonumber
          \end{equation}
          with  $\sigma^k$ the three Pauli matrices ($k$ running from 1 to 3). Also
          $ \lbrace \gamma^{\mu}, \gamma^{\nu} \rbrace= 2 g^{\mu \nu}$ and $\lbrace \gamma_5,\gamma^{\mu} \rbrace =0$, with $\mu$ and $\nu$ running from 0 to 3. The metric tensor is $g_{00}=+1$, $g_{kk}=-1$, $g_{\mu \neq \nu}=0$.

\end{itemize}

