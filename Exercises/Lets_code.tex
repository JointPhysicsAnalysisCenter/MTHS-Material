The following practical exercise requires you to have a workin python installation, together with numpy, scipy, matplotlib, and iminuit.\\
We are gonna perform correlated fits to montecarlo sampled 2-pt correlators, calculated from Wilson lines for a $L=32$ lattice with periodic boundary conditions.\\
\begin{itemize}
\item Firt, start by downloading the folder Code/sheet\_7\_code from within the school's github repo \href{https://github.com/JointPhysicsAnalysisCenter/MTHS-Material}{here}
\begin{enumerate}
    \item \textbf{Install Python in a Conda Environment with Dependencies}
    
    Make sure you have \texttt{conda} installed. Easiest way for homebrew users is ``brew install --cask anaconda''. For linux, follow these instructions \href{https://docs.conda.io/projects/conda/en/latest/user-guide/install/linux.html}{here}

    Create a new conda environment and install the required packages:
    \begin{verbatim}
    conda create --name lattice-qcd python=3.xxx
    conda activate lattice-qcd
    conda install numpy scipy matplotlib iminuit jupyterlab
    \end{verbatim}
    
    \item \textbf{Start Jupyter Lab and Open the Notebook}
    
    Launch Jupyter Lab from the terminal:
    \begin{verbatim}
    jupyter lab
    \end{verbatim}
    In your web browser, navigate to the Jupyter Lab interface. Open the notebook file inside the folder you just downloaded.
    
    \item \textbf{Jupyter Notebook Cheat Sheet}
    
    For basic commands and shortcuts in Jupyter Notebook, refer to the [Jupyter Notebook Cheat Sheet](https://jupyter-notebook.readthedocs.io/en/stable/notebook.html).
    \end{enumerate}
\item Once you do so, open the .ipynb inside. Note that the first two cells just load packages and fixed variables, described in the notebook.
\item Now, follow the instructions in the notebook to try and obtain a good set of correlated Jackknife fits to the correlaors provided in the folder.
\end{itemize}