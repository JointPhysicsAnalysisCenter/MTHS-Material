
Given the Omn\`{e}s function:
\begin{equation}
    \Omega(s)=\exp\left[\frac{s}{\pi}\int_{4m_{\pi}^{2}}^{\infty}dz\frac{\delta(z)}{z(z-s)}\right]\,,
    \label{Eq:Omnes}
\end{equation}

\begin{enumerate}
    \item Consider the low-energy expansion of the pion form factor $F_{\pi}(s)\equiv\Omega(s)$:
          \begin{equation}
              F_{\pi}(s)=1+\frac{1}{6}\langle r^{2}_{\pi}\rangle\,s+\mathcal{O}(s^{2})\,,
          \end{equation}
          and deduce the sum rule:
          \begin{equation}
              \langle r^{2}_{\pi}\rangle=\frac{6}{\pi}\int_{4m_{\pi}^{2}}^{\infty}dz\frac{\delta(z)}{z^{2}}\,.
          \end{equation}
          The quantity $\sqrt{\langle r^{2}_{\pi}\rangle}$ is called charge radius of the pion, see \href{https://pdglive.lbl.gov/DataBlock.action?node=S008CR}{PDF} for summary of the experimental measurements.
    \item Assume that the phase shift $\delta(s)$ reaches $k\pi$ ($k$ is an integer) at $s=\Lambda^{2}$ and stays at that value for larger $s$. What is the behavior of $\Omega(s)$ in the limit $|s|\to\infty$?\\
    \item What is the resulting function $\Omega(s)$ for an infinitely narrow resonance, {\it{i.e.}} consider \mbox{$\delta(s)=\pi\theta(s-M^{2})$}?
\end{enumerate}
