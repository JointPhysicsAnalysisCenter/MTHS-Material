Using the following transformation properties for spinors under charge and parity conjugation
\begin{equation*}
\begin{aligned}
& \psi\left(\vec{x} , t\right) \xrightarrow{\mathcal{P}} \psi\left(\vec{x} , t\right)^{\mathcal{P}}=\gamma_4 \psi\left(-\vec{x} , t\right)  \quad\quad \bar{\psi}\left(\vec{x} , t\right) \xrightarrow{\mathcal{P}} \bar{\psi}\left(\vec{x} , t\right)^{\mathcal{P}}=\bar{\psi}\left(-\vec{x} , t\right) \gamma_4,
\end{aligned}
\end{equation*}
\begin{equation*}
\begin{aligned}
& \psi(x) \xrightarrow{\mathcal{C}} \psi(x)^{\mathcal{C}}=C^{-1} \bar{\psi}(x)^T  \quad \quad \bar{\psi}(x) \xrightarrow{\mathcal{C}} \bar{\psi}(x)^{\mathcal{C}}=-\psi(x)^T C \quad\quad C \gamma_\mu C^{-1}=-\gamma_\mu^T.
\end{aligned}
\end{equation*}
Prove that the $\pi^+$ operator seeing in the lectures does indeed behave as expected. What about a $\rho$-type meson operator? \\
