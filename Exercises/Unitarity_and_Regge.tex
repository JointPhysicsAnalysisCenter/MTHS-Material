Van Hove proposed a physically intuitive picture of a Reggeon by relating it to Feynman diagrams in the cross-channels. We will explore this picture of Reggeization with a simple model.

\begin{enumerate}
    \item \textbf{Elementary $t$-channel exchanges} \\
          Consider the amplitude corresponding to a particle with spin-$J$ and mass $m_J$  exchanged in the $t$-channel as:
          %%
          \begin{equation}
              \label{eq:AJ}
              A^J(s,t) = i \, g_J \, \left( q_1^{\mu_1} \dots q_1^{\mu_J} \right) \frac{P^J_{\mu_1\dots\mu_J,\nu_1\dots\nu_J}(k)}{m_J^2 - t} \left( q_{\bar{2}}^{\nu_1} \dots q_{\bar{2}}^{\nu_J} \right)
          \end{equation}
          %%
          where $g_J$ is a coupling constant with dimension $2-2J$ (i.e., $A^J(s,t)$ is dimensionless) and the projector of spin-$J$ is defined from the polarization tensor of rank-$J\geq1$ as
          %%
          \begin{equation}
              P^J_{\mu_1\dots\mu_J,\nu_1\dots\nu_J}(k) = \frac{(J+1)}{2} \, \sum_{\lambda}
              \epsilon^{\mu_1\dots\mu_J}(k,\lambda) \, \epsilon^{*\nu_1\dots\nu_J}(k,\lambda) ~.
          \end{equation}
          %%

          Using the exchange momentum $k = q_1 + q_{\bar{3}} = q_1 - q_3$, calculate the amplitudes corresponding to $J=0,1,2$ exchanges in terms of $t= k^2$, the modulus of 3-momentum and cosine of scattering angle in the $t$-channel frame, $q_t$ and $\cos\theta_t$ respectively. Use the explicit forms of the projectors:
          %%
          \begin{gather}
              P^0(k^2) = 1 \\
              P^1_{\mu\nu}(k^2) \equiv \tilde{g}_{\mu\nu} =   \frac{ k_\mu \, k_\nu}{ k^2} - g_{\mu\nu}\\
              P^2_{\mu\nu\alpha\beta}(k^2) = \frac{3}{4}\left(\tilde{g}_{\mu\alpha} \, \tilde{g}_{\nu\beta} + \tilde{g}_{\mu\beta} \, \tilde{g}_{\nu\alpha}\right) - \frac{1}{2} \, \tilde{g}_{\mu\nu} \, \tilde{g}_{\alpha\beta} ~,
          \end{gather}
          %%
          and conjecture a generalization of the amplitude for arbitrary integer $J$.

          \noindent\textit{Hint: Show that in the $t$-channel frame, the exchange particle is at rest and therefore $\tilde{g}_{\mu\nu}$ reduces to a $\delta_{ij}$ with respect to only spacial momenta.}

          \begin{solution}

              \noindent We start by considering the elastic scattering of two identical, spinless particles with 4-momentum $q_{i}$ and with mass $q_i^2 = m^2$.
              We define the usual Mandelstam variables
              %%
              \begin{gather}
                  s = (q_1 + q_2)^2 = (q_3 + q_4)^2 \nonumber\\
                  t = (q_1 - q_3)^2 = (q_4 - q_2)^2 \nonumber \\
                  u = (q_1 - q_4)^2 = (q_2 - q_3)^2 \nonumber
              \end{gather}
              %%
              We refer to the $s$-channel as the physical region describing the process
              %%
              \begin{equation}
                  1 \, (q_1) + 2 \, (q_2) \to 3 \, (q_3) + 4 \, (q_4) ~, \nonumber
              \end{equation}
              %%
              while in the $t$-channel we consider
              \begin{equation}
                  1 \, (q_1) + \bar{3} \, (q_{\bar{3}})
                  \to  \bar{2} \, (q_{\bar{2}}) + 4 \, (q_4) ~. \nonumber
              \end{equation}
              %%

              The $J=0$ is trivial
              \begin{equation}
                  A^0(s,t) = i \, g_0 \, \frac{1}{m_0^2 - t} = i \, g_0 \, \frac{P_0(\cos\theta_t)}{m_0^2 - t} ~.
              \end{equation}
              For $J=1$ use $q_1 = (\sqrt{t}/2, q_t \, \hat{z})$ and $q_{\bar{3}} = (\sqrt{t}/2, - q_t \, \hat{z} )$.
              In the $t$-channel CM frame we have $k = (q_1 - q_3) = (q_1 + q_{\bar{3}}) = (\sqrt{t}, \vec{0})$ and
              %%
              \begin{equation}
                  -\tilde{g}_{\mu\nu} =  - g_{\mu\nu} + \frac{k_\mu \, k_\nu}{t}
                  =  -\left[\delta_{\mu0} \, \delta_{\nu0} - \delta_{ij} \right] + \frac{\sqrt{t}^2}{t} \, \delta_{\mu0} \, \delta_{\nu0} = +\delta_{ij} ~.
              \end{equation}
              %%
              thus $q_1^\mu \, \tilde{g}_{\mu\nu} q_{\bar{2}}^\nu =  \vec{q}_1 \, \cdot \, \vec{q}_{\bar{2}} = q_t^2 \, \cos\theta_t$. Similarly $q_1^\mu \, \tilde{g}_{\mu\nu} q_{1}^\nu =  q_{\bar{2}}^\mu \, \tilde{g}_{\mu\nu} q_{\bar{2}}^\nu  = q_t^2$ and we have
              %
              \begin{equation}
                  A^1(s,t) = i g_1 \, q_t^2 \frac{ \cos\theta_t}{m_1^2 - t} = i g_1 \, q_t^2 \frac{P_1(\cos\theta_t)}{m_1^2 - t} ~,
              \end{equation}
              %%
              and finally also
              %%
              \begin{equation}
                  A^2(s,t) = i g_2 \, q_t^4 \,\frac{\frac{1}{2}(3 \cos\theta_t - 1)}{m_2^2 - t} =  i g_2 \, q_t^4 \, \frac{ P_2(\cos\theta_t)}{m_2^2 - t} ~.
              \end{equation}
              %%
              The generalization to arbitrary $J$ is
              %%
              \begin{equation}
                  A^J(s,t) = ig_J \, q_t^{2J} \, \frac{ P_J(\cos\theta_t)}{m_J^2 - t} ~.
              \end{equation}
              %%
          \end{solution}

    \item \textbf{Unitarity vs Elementary exchanges} \\
          Express the amplitude entirely in terms of invariants $s$ and $t$. Use the optical theorem to relate the elastic amplitude to a total hadronic cross section:
          \begin{equation}
              \sigma_\text{tot} = \frac{1}{2q\sqrt{s}} \, \Im A^J(s,t=0)~.
          \end{equation}
          Unitarity (via the Froissart-Martin bound) prohibits $\sigma_\text{tot}$ from growing faster than $\log^2 s$ as $s\to\infty$~. What is then the maximal spin a single elementary exchange can have while satisfying this bound? Why is this a problem?

          \begin{solution}
              We have
              %%
              \begin{equation}
                  \sigma_\text{tot} = \left. \frac{1}{2q\sqrt{s}} \, \frac{g_J}{m_J^2} \, q_t^{2J} \, P_J(\cos\theta_t)  \right |_{t=0} ~.
              \end{equation}
              %%
              We have $q_t^2 \, \cos\theta_t = (s-u)/ 4$ so that as $s\to \infty$, we have:
              %%
              \begin{equation}
                  \sigma_\text{tot} \sim s^{J-1} ~.
              \end{equation}
              %%
              To satisfy the Froissart bound, the maximally allowed spin then is $J=1$.
          \end{solution}

    \item \textbf{Van Hove Reggeon} \\
          Consider an amplitude of the form
          %%
          \begin{equation}
              \label{eq:vanhove}
              A(s,t) = \sum_{J=0}^\infty g \, r^{2J} \, \frac{ (q_t^2 \, \cos\theta_t)^J}{J- \alpha(t)} ~.
          \end{equation}
          %%
          Here $\alpha(t) = \alpha(0) + \alpha^\prime \, t$ is a real, linear Regge trajectory, $g$ is a dimensionless coupling constant and $r \sim 1$ fm is a range parameter.
          Compare Eq.~\ref{eq:vanhove} with Eq.~\ref{eq:AJ}, write the mass of the $J$th pole, $m_J^2$, as a function of the Regge parameters $\alpha(0)$ and $\alpha^\prime$. Interpret the pole structure in terms of the spectrum of particles in the model.
          \\

          \noindent If the sum is truncated to a finite $J_\text{max}$, and we take the $s\to \infty$ limit, what is the high energy behavior of the amplitude?

          \begin{solution}
              We can write
              %%
              \begin{equation}
                  J-\alpha(t) = J - \alpha(0) - \alpha^\prime \, t = \alpha^\prime \left( (J-\alpha(0))/\alpha' - t\right)
              \end{equation}
              %5
              and thus we have $m_J^2 = (J-\alpha(0))/\alpha^\prime$.

              We can use
              %%
              \begin{align}
                  (\cos\theta_t)^J & = \sum_{J+J^\prime \text{ even}}^J \frac{(J+1)!}{(J-J')!! \, (J+J'+1)!!} \, P_{J'}(\cos\theta_t) \\
                                   & = \sum_{J+J^\prime \text{ even}}^J \mu_{JJ'} \, P_{J'}(\cos\theta_t)
              \end{align}
              %%
              to write
              %%
              \begin{align}
                  A(s,t) & = \sum_J \, \sum_{J'=0}^J \left(\frac{g \, r^{2J} \mu_{JJ'}}{\alpha'}\right) q_t^{2J} \, \frac{P_{J'}(\cos\theta_t)}{m_J^2 - t} ~,
                  \\
                         & = \sum_J \, \sum_{J'=0}^J g_{JJ'} \, q_t^{2J} \, \frac{P_{J'}(\cos\theta_t)}{m_J^2 - t} ~,
              \end{align}
              %%
              Comparing with the form of our elementary exchanges, this amplitude is an infinite sum of particles with spin-$J$ and mass $m_J^2$ but also all same parity daughters at the same mass.

              If the sum is truncated at $J_\text{max}$ the $s\to\infty$ limit is dominated by the largest spin exchange and we have $A_\text{trunc}(s,t) \propto s^{J_\text{max}}$.
          \end{solution}

    \item \textbf{Analytic continuation in $J$} \\
          Show that if the summation is kept infinite, the amplitude can be re-summed to something that is entirely analytic in $s$, $t$, $u$, and $J$.

          \noindent \textit{Hint: Use the Mellin transform
              %%
              \begin{equation}
                  \frac{1}{J-\alpha(t)} = \int_0^1 dx \, x^{J-\alpha(t) - 1} ~,
              \end{equation}
              %%
              to express the amplitude in terms of the Gaussian hypergeometric function and the Euler Beta function
              %%
              \begin{equation}
                  B(b, c-b) \, _2F_1(1, b, c; z) =  \int_0^1 dx \, \frac{ x^{b-1} \, (1-x)^{c-b-1}}{1-x \, z} ~.
              \end{equation}
              %%
          }

          \begin{solution}
              Go back to the original form in terms of monomials, we can write
              %%
              \begin{equation}
                  A(s,t) = \sum_{J=0} \int_0^1 dx \, g \, r^{2J} (q_t^2 \, \cos\theta_t)^J \, x^{J-\alpha(t)-1} ~.
              \end{equation}
              %%  
              Collecting all things with powers of $J$, we notice a geometric series which can be summed analytically
              %%
              \begin{equation}
                  A(s,t) = g \,\int_0^1 dx \, \frac{x^{-\alpha(t) -1}}{1 - r^2 \, q_t^2 \, \cos\theta_t \, x} ~.
              \end{equation}
              %%
              Comparing with the definition of the hypergeometric function, we can identify $z = r^2 \, q_t^2 \, \cos\theta_t$ and $b = -\alpha(t)$. Since there is no $(1-x)$ term we require $c = b+1= 1-\alpha(t)$. Thus we have
              %5
              \begin{align}
                  A(s,t) & = \frac{ \Gamma(-\alpha(t))}{\Gamma(1-\alpha(t))} \, _2F_1\left(1,-\alpha(t), 1-\alpha(t), (q_t \,r)^2 \, \cos\theta_t \right)
                  \\
                         & = \Gamma(-\alpha(t)) \,  _2\tilde{F}_1\left(1,-\alpha(t), 1-\alpha(t), (q_t \,r)^2 \, \cos\theta_t \right) ~.
              \end{align}
              %%
          \end{solution}

    \item \textbf{Unitarity vs Reggeized exchanges} \\
          Revisit $\mathbf{b)}$ with the resummed amplitude. Take the $s\to\infty$ limit and set a limit on the maximal intercept $\alpha(0)$ which is allowed by unitarity.

          \noindent \textit{Hint: Assume that $\alpha(0) > -1$ and use the asymptotic behavior of the hypergeometric function given by
              %%
              \begin{equation}
                  _2F_1(1,b,c;z) \to \frac{\Gamma(c)\,\Gamma(1-b)}{\Gamma(1) \,\Gamma(c-b)} \, (-z)^{-b}  ~.
              \end{equation}
              %%
          }

          \begin{solution}
              From the hypergeometric form we can take $s\to \infty$ which takes $q_t^2 \, \cos\theta_t = (s-u)/4 \to \infty$ and we can write
              %%
              \begin{equation}
                  A(s,t) = g_0 \, \Gamma(-\alpha(t)) \, \Gamma(1+\alpha(t)) \, \left(\frac{u-s}{4r^{-2}}\right)^{\alpha(t)} ~.
              \end{equation}
              %%
              So, we have
              %%
              \begin{equation}
                  \Im A(s,0) \propto \Im (-s)^{\alpha(0)} \propto \sin \pi \alpha(0) \, s^{\alpha(0)}
              \end{equation}
              %%
              so $\sigma_\text{tot} \sim s^{\alpha(0)-1}$ and unitarity requires $\alpha(0) \leq 1$.
          \end{solution}

    \item \textbf{The Reggeon ``propagator"} \\
          Modify Eq.~\ref{eq:AJ} to have a definite signature by defining
          %%
          \begin{equation}
              A^{\pm}(s,t) = \frac{1}{2}\left[A(s,t) \pm A(u,t) \right]~.
          \end{equation}
          %%
          Repeat $\textbf{d)}$ and  $\textbf{e)}$ with this signatured amplitude. Compare with the canonical form of the Reggeon exchange:
          %%
          \begin{equation}
              A^\pm_\mathbb{R}(s,t) = \beta(t) \, \frac{1}{2}\left[\pm1 + e^{-i\pi\alpha(t)}\right] \, \Gamma(-\alpha(t)) \, \left(\frac{s}{s_0}\right)^{\alpha(t)} ~.
          \end{equation}
          %%
          Identify the Regge residue $\beta(t)$ and characteristic scale $s_0$ in terms of the parameters $g_0$ and $r$.

          \begin{solution}
              As we see above, switching $s\leftrightarrow u$ introduces a minus sign and we have
              %%
              \begin{equation}
                  A^\pm(s,t) = g_0 \, \Gamma(1+\alpha(t)) \, \frac{1}{2}[\pm1 + e^{-i\pi\alpha(t)}] \, \Gamma(-\alpha(t)) \left(\frac{s-u}{4r^{-2}}\right)^{\alpha(t)} ~,
              \end{equation}
              %%
              and we can read off $\beta(t) = g_0 \, \Gamma(1+\alpha(t))$. Using $s \sim -u$ we also see $s_0 = 2 \, r^{-2}$
          \end{solution}

\end{enumerate}