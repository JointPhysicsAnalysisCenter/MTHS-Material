The quintessential dual amplitude was first proposed by Veneziano for $\omega\to3\pi$ and later applied to elastic $\pi\pi$ scattering by Shapiro and Lovelace. Consider the $\pi^+\pi^-$ scattering amplitude of the form
%%
\begin{equation}
    \label{eq:Astu}
    \mathcal{A}(s,t,u) = V(s,t) + V(s,u) - V(t,u) ~.
\end{equation}
%%
with each
%%
\begin{equation}
    \label{eq:Fst}
    V(s,t) = \frac{\Gamma(1-\alpha(s)) \, \Gamma(1-\alpha(t))}{\Gamma(1-\alpha(s) - \alpha(t))} ~,
\end{equation}
%%
where $\alpha(s) = \alpha(0) + \alpha^\prime s$ is a real, linear Regge trajectory with $\alpha^\prime > 0$.

\begin{enumerate}
    \item \textbf{Duality} \\
          Show that the function $V(s,t)$ is symmetric in $s\leftrightarrow t$ and dual, i.e., it can be written entirely as a sum of either $s$-channel poles OR $t$-channel poles but never both simultaneously. Compare with the Reggeized amplitude in the previous problem, was that amplitude dual?

          \noindent \textit{Hint: Relate $V(s,t)$ to the Euler Beta function
              \begin{equation}
                  B(x,y) = \frac{\Gamma(x) \, \Gamma(y)}{\Gamma(x+y)}
              \end{equation}
              and use the identities $B(x,y) = B(y,x)$ and
              %%
              \begin{equation}
                  B(p-x, q-y) = \sum_{J=1}^\infty \, \frac{\Gamma(J-p+1 +x)}{\Gamma(J) \, \Gamma(-p + 1 +x)} \, \frac{1}{J-1+q-y} ~.
              \end{equation}
              %%
          }

          \begin{solution}
              We can write:
              %%
              \begin{equation}
                  V(s,t) = (1-\alpha(s) - \alpha(t)) \, B(1-\alpha(s), 1-\alpha(t)) ~.
              \end{equation}
              %%
              Then using the expansion of the Beta function on its first argument, we have
              %%
              \begin{equation}
                  V(s,t) = (1-\alpha(s) - \alpha(t)) \, \sum_{J=1}^\infty \, \frac{\Gamma(J-1+\alpha(t))}{\Gamma(J) \, \Gamma(\alpha(t))} \, \frac{1}{J-\alpha(s)}
              \end{equation}
              %%
              which only has poles in $\alpha(s)$. Because of the $s\leftrightarrow t$ symmetry, we can write the exact same expression with only poles in $\alpha(t)$.
          \end{solution}

    \item \textbf{Isospin basis} \\
          Define the $s$-channel isospin basis through
          %%
          \begin{equation}
              \begin{pmatrix}
                  \mathcal{A}^{(0)}(s,t,u) \\
                  \mathcal{A}^{(1)}(s,t,u) \\
                  \mathcal{A}^{(2)}(s,t,u)
              \end{pmatrix}
              =
              \frac{1}{2}\begin{pmatrix}
                  3 & 1 & 1  \\
                  0 & 1 & -1 \\
                  0 & 1 & 1
              \end{pmatrix}
              \begin{pmatrix}
                  \mathcal{A}(s,t,u) \\
                  \mathcal{A}(t,s,u) \\
                  \mathcal{A}(u,t,s)
              \end{pmatrix}
              ~.
          \end{equation}
          %%
          Write down the definite-isospin amplitudes in terms of $V$'s. Comment on the symmetry properties of each isospin amplitude with respect to $t \leftrightarrow u$.

          \begin{solution}
              We have:
              %%
              \begin{gather}
                  \mathcal{A}^{(0)}(s,t,u) = \frac{1}{2} \left[ 3 \, V(s,t) + 3\,V(s,u) - V(t,u)\right] \\
                  \mathcal{A}^{(1)}(s,t,u) = V(s,t) - V(s,u) \\
                  \mathcal{A}^{(2)}(s,t,u) = V(t,u) ~.
              \end{gather}
              %%
              $I=0,2$ are symmetric in $t\leftrightarrow u$ while $I=1$ is anti-symmetric as required by Bose symmetry.
          \end{solution}

    \item \textbf{Chew-Frautshi plot} \\
          Locate where each $\mathcal{A}^{(I)}(s,t,u)$ will have poles in the $s$-channel physical region. What is their residue? Draw a schematic Chew-Frautschi plot of the resonance spectrum in each isospin channel.

          \begin{solution}
              A single $V(s,t)$ will have poles at all $\alpha(s) = J\geq 1$ and all possible daughters. The residues are
              %%  
              \begin{equation}
                  - (J-1+\alpha(t)) \, \frac{\Gamma(J-1+\alpha(t))}{\Gamma(J) \, \Gamma(\alpha(t))} = \frac{-1}{\Gamma(J)} \frac{\Gamma(J+\alpha(t))}{\Gamma(\alpha(t))} = - \frac{(\alpha(t))_J}{(J-1)!}.
              \end{equation}
              %%
              For a linear trajectory this is a order $J$ polynomial in $t$ and therefore in $z_s$.

              The symmetry factors in $I=0,1$ will remove all odd (even) $J$ daughters. The $I=2$ amplitude has no $s$ dependence and therefore no isospin-2 poles at all.
          \end{solution}

    \item \textbf{Regge limit} \\
          Now consider the limit $t \to \infty$ and $u \to - \infty$ with $s \leq 0$ is fixed. What is the asymptotic behavior of $V(s,t)$ and $V(s,u)$? Assume that $V(t,u)$ vanishes faster than any power of $s$ in this limit. What is the resulting behavior of the isospin amplitudes $\mathcal{A}^{(I)}(s,t,u)$ in this limit?

          \noindent \textit{ Hint: Use the Sterling approximation of the $\Gamma$ function., i.e. as $|x|\to \infty$
              \begin{equation}
                  \Gamma(x) \to \sqrt{\frac{2\pi}{x}} \left( \frac{x}{e} \right)^x ~.
              \end{equation}
          }

          \begin{solution}
              Starting with
              \begin{equation}
                  V(s,t)  \to \Gamma(1-\alpha(s)) \, (-\alpha(t))^{\alpha(s)} \sim \Gamma(1-\alpha(s)) \, \left(-\alpha^\prime \, t\right)^{\alpha(s)} ~.
              \end{equation}
              Similarly
              \begin{equation}
                  V(s,u)  \to \Gamma(1-\alpha(s)) \, (-\alpha(u))^{\alpha(s)} \sim \Gamma(1-\alpha(s)) \, \left(-\alpha^\prime \, t\right)^{\alpha(s)} ~.
              \end{equation}
              Thus the combination
              %%
              \begin{align}
                  V(s,t) \pm V(s,u) & \to \Gamma(1-\alpha(s)) \times \left[\left(-\alpha^\prime \, t\right)^{\alpha(s)} \pm \left(-\alpha^\prime \, u\right)^{\alpha(s)} \right] \\
                                    & = \Gamma(1-\alpha(s)) \times \left[1\pm e^{-i\pi\alpha(s)}\right] \, \left(\alpha^\prime \, t\right)^{\alpha(s)}
              \end{align}
              %%
          \end{solution}

    \item \textbf{Ancestors and Strings}\\
          Consider the model now with a complex trajectory $\alpha(s) = a_0 + \alpha^\prime \, s + i \,\Gamma$ with $\Gamma > 0$ to move the poles off the real axis. Reexamine the the Chew-Frautshi plot for the $I=1$ amplitude using this trajectory, why is the resulting spectrum problematic? Try a real but non-linear trajectory, say $\alpha(s) = \alpha_0 + \alpha^\prime \, s + \alpha^{\prime\prime} \, s^2$, what is the spectrum like now?

          Compare the requirements of the trajectory for $V(s,t)$ to make sense with the energy levels of a rotating relativistic string with a string tension $T$:
          %%
          \begin{equation}
              E_J^2 = \frac{1}{2\pi \, T} \, J ~.
          \end{equation}
          %%
          What is a possible microscopic picture of hadrons if the Veneziano amplitude is believed?

          \begin{solution}
              If we allow $\alpha(t)$ to be complex, then at a pole $\alpha(s) \to J + i \Gamma$ and the residue we calculated
              \begin{equation}
                  \frac{\Gamma(J+i \,\Gamma + \alpha(t))}{\Gamma(\alpha(t))} ~,
              \end{equation}
              is no longer a fixed order polynomial in $t$. It will thus give contributions to ALL spins at each pole, i.e. introduce an infinite number of ancestors. Similarly if $\alpha(s)$ is non-linear, we will have finitely many ancestors but still unphysical poles nonetheless.

              This means the Veneziano amplitude \textit{only} gives a physical picture for real and linear trajectories. This means we require $J \propto s \sim m^2$ which mimics the spectrum of states in a relativistic rotating string. This gives rise to the stringy picture of a $q\bar{q}$ pair connected by a gluon flux tube and later the entire field of string theories.
          \end{solution}

\end{enumerate}