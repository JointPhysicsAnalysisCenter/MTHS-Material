The quintessential dual amplitude was first proposed by Veneziano for $\omega\to3\pi$ and later applied to elastic $\pi\pi$ scattering by Shapiro and Lovelace. Consider the $\pi^+\pi^-$ scattering amplitude of the form
%%
    \begin{equation}
        \label{eq:Astu}
        \mathcal{A}(s,t,u) = V(s,t) + V(s,u) + V(t,u) ~.
    \end{equation}
%%
with each 
%%
    \begin{equation}   
        \label{eq:Fst}
        V(s,t) = \frac{\Gamma(1-\alpha(s)) \, \Gamma(1-\alpha(t))}{\Gamma(1-\alpha(s) - \alpha(t))} ~,
    \end{equation}
%%
where $\alpha(s) = \alpha(0) + \alpha^\prime s$ is a real, linear Regge trajectory with $\alpha^\prime > 0$. 

\subsection{Duality} 
Show that the function $V(s,t)$ is symmetric in $s\leftrightarrow t$ and dual, i.e., it can be written entirely as a sum of either $s$-channel poles OR $t$-channel poles but never both simultaneously. Compare with the Reggeized amplitude in the previous problem, was that amplitude dual?

\noindent \textit{Hint: Relate $V(s,t)$ to the Euler Beta function
    \begin{equation}
        B(x,y) = \frac{\Gamma(x) \, \Gamma(y)}{\Gamma(x+y)}
    \end{equation}
and use the identities $B(x,y) = B(y,x)$ and
%%
    \begin{equation}
        B(p-x, q-y) = \sum_{J=1}^\infty \, \frac{\Gamma(J-p+x)}{\Gamma(J) \, \Gamma(-p + 1 +x)} \, \frac{1}{J-1+q-y} ~.
    \end{equation}
%%
}

\subsection{Isospin basis} 
Define the $s$-channel isospin basis through
%%
    \begin{equation}
        \begin{pmatrix}
            \mathcal{A}^{(0)}(s,t,u) \\
            \mathcal{A}^{(1)}(s,t,u) \\
            \mathcal{A}^{(2)}(s,t,u)             
        \end{pmatrix}
        = 
        \begin{pmatrix}
            3 & 1 & 1 \\
            0 & 1 & -1 \\
            0 & 1 & 1 
        \end{pmatrix}
        \begin{pmatrix}
            \mathcal{A}(s,t,u) \\
            \mathcal{A}(t,s,u) \\
            \mathcal{A}(u,t,s)              
        \end{pmatrix}
        ~.
    \end{equation}
%%
Write down the definite-isospin amplitudes in terms of $V$'s. Comment on the symmetry properties of each isospin amplitude with respect to $t \leftrightarrow u$. 

\subsection{Chew-Frautshi plot}
Locate where each $\mathcal{A}^{(I)}(s,t,u)$ will have poles in the $s$-channel physical region. Draw a schematic Chew-Frautschi plot of the resonance spectrum in each isospin channel.

\subsection{Regge limit}
Now consider the limit $s \to \infty$ and $u \to - \infty$ with $t \leq 0$ is fixed. What is the behavior of $V(s,t)$ and $V(s,u)$ in this limit? Do the same for $V(t,u)$ additionally assuming that $\alpha(s)$ is complex with $\Im\alpha(s) /\log s \to \infty$ asymptotically. What is the resulting behavior of the isospin amplitudes $\mathcal{A}^{(I)}(s,t,u)$ in this limit?

\noindent \textit{ Hint: Use the Sterling approximation of the $\Gamma$ function., i.e. as $|x|\to \infty$
    \begin{equation}
        \Gamma(x) \to \sqrt{\frac{2\pi}{x}} \left( \frac{x}{e} \right)^x ~.
    \end{equation}
}

\subsection{Ancestors and Strings}
Consider the model now with a complex trajectory $\alpha(s) = a_0 + \alpha^\prime \, s + i \gamma \sqrt{s-s_\text{th}}$ with $\gamma >0$ to move the poles off the real axis. Reexamine the the Chew-Frautshi plot for the $I=1$ amplitude using this trajectory, why is the resulting spectrum problematic? Try a real but non-linear trajectory, say $\alpha(s) = \alpha_0 + \alpha^\prime \, s + \alpha^{\prime\prime} \, s^2$, what is the spectrum like now?

Compare the requirements of the trajectory for $V(s,t)$ to make sense with the energy levels of a rotating relativistic string with a string tension $T$:
%%
    \begin{equation}
        E_J^2 = \frac{1}{2\pi \, T} \, J ~.
    \end{equation}
%%
What is a possible microscopic picture of hadrons in the Veneziano amplitude is believed? 
